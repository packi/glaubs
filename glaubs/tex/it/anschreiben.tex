\documentclass[a4paper,10pt,parskip=half,italian]{scrlttr2}
\usepackage[utf8]{inputenc}
\usepackage{babel}
\usepackage{fullpage}
\usepackage{graphicx}

\LoadLetterOption{SN}
\setlength{\parindent}{0pt}
\KOMAoptions{foldmarks=hb}
\makeatletter
\@setplength{toaddrvpos}{55mm}
\@setplength{toaddrhpos}{120mm}
\@setplength{toaddrheight}{25mm}
\@setplength{sigbeforevskip}{0mm}
\@setplength{firstheadwidth}{\textwidth}
\makeatother

\renewcommand*{\raggedsignature}{\raggedright}

\begin{document}

\begin{letter}{ {{ recipient }} }

\setkomavar{fromname}{Referendum Stop BÜPF}
\setkomavar{fromaddress}{Röschibachstrasse 26 \\ 8037 Zürich \\ Tel. 079 7754555 \\E-Mail: ok@stopbuepf.ch}

\setkomavar{backaddress}{Referendum Stop BÜPF, 8037 Zürich}

\setkomavar{subject}{Attestazione del diritto di voto: Referendum federale contro la legge federale sulla sorveglianza della corrispondenza postale e del traffico delle telecomunicazioni (LSCPT)}

\setkomavar{signature}{
\includegraphics[width=5cm]{sigJAN} \\
\vspace{3mm}
COMITATO DEL REFERENDUM FEDERALE \\
Referendum Stop BÜPF \\
Jorgo Ananiadis
}

\opening{Gentili Signore e Signori,}

visti gli articoli 62 e 63 della legge federale del 17 dicembre 1976 sui diritti politici vi inviamo in allegato {{ listCount }} liste con firme (No {{ listMin }} - {{ listMax }}) a sostegno del nostro referendum federale contro la

\leftskip=3mm
Legge federale sulla sorveglianza della corrispondenza postale e del traffico delle telecomunicazioni (LSCPT)

\leftskip=0mm
sulle quali figurano complessivamente {{ sigCount }} firme. Vi preghiamo cortesemente di attestare il diritto di voto dei firmatari e di rinviarci le liste con le relative attestazioni entro il {{ dueDate }} al più tardi, al seguente indirizzo:

\leftskip=3mm
Referendum Stop BÜPF \\
Röschibachstrasse 26 \\
8037 Zürich

\leftskip=0mm
\closing{Ringraziandovi già sin d’ora per il vostro sostegno, vi preghiamo di gradire i nostri distinti saluti.}

\end{letter}
\end{document}

