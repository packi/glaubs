\documentclass[a4paper,10pt,parskip=half]{scrlttr2}
\usepackage[utf8]{inputenc}
\usepackage[nswissgerman]{babel}
\usepackage{fullpage}
\usepackage{graphicx}

\LoadLetterOption{SN}
\setlength{\parindent}{0pt}
\KOMAoptions{foldmarks=hb}
\makeatletter
\@setplength{toaddrvpos}{55mm}
\@setplength{toaddrhpos}{120mm}
\@setplength{toaddrheight}{25mm}
\@setplength{sigbeforevskip}{0mm}
\@setplength{firstheadwidth}{\textwidth}
\makeatother

\renewcommand*{\raggedsignature}{\raggedright}

\begin{document}

\begin{letter}{ {{ recipient }} }

\setkomavar{fromname}{Referendum Stop BÜPF}
\setkomavar{fromaddress}{Röschibachstrasse 26 \\ 8037 Zürich \\ Tel. 079 7754555 \\E-Mail: ok@stopbuepf.ch}

\setkomavar{backaddress}{Referendum Stop BÜPF, 8037 Zürich}

\setkomavar{subject}{Stimmrechtsbescheinigung: Eidgenössisches Referendum gegen das Bundesgesetz betreffend die Überwachung des Post- und Fernmeldeverkehrs (BÜPF) }

\setkomavar{signature}{
\includegraphics[width=5cm]{sigJAN} \\
\vspace{3mm}
Eidg. Referendumskomitee \\
Referendum Stop BÜPF \\
Jorgo Ananiadis
}

\opening{Sehr geehrte Damen und Herren }

Gestützt auf die Artikel 62 und 63 des Bundesgesetzes vom 17. Dezember 1976 über die politischen Rechte stellen wir Ihnen in der Beilage {{ listCount }} Unterschriftenlisten (Nrn. {{ listMin }} - {{ listMax }}) unseres eidgenössischen Referendums gegen das

\leftskip=3mm
Bundesgesetz betreffend die Überwachung des Post- und Fernmeldeverkehrs (BÜPF)

\leftskip=0mm
mit insgesamt {{ sigCount }} Unterschriften zu.

Wir ersuchen Sie höflich, das Stimmrecht der Unterzeichnerinnen und Unterzeichner zu bescheinigen.

Dürfen wir Sie darum bitten, die Unterschriftenlisten bis spätestens am 4. Juli per A-Post zurückzusenden an:

\leftskip=3mm
Referendum Stop BÜPF \\
Röschibachstrasse 26 \\
8037 Zürich

\leftskip=0mm
\closing{Für Ihre Bemühungen danken wir Ihnen im Voraus verbindlich und grüssen Sie freundlich.}

\end{letter}
\end{document}

