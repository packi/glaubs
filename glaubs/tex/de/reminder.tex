\documentclass[a4paper,10pt,parskip=half]{scrlttr2}
\usepackage[utf8]{inputenc}
\usepackage[nswissgerman]{babel}
\usepackage{fullpage}
\usepackage{graphicx}

\LoadLetterOption{SN}
\setlength{\parindent}{0pt}
\KOMAoptions{foldmarks=hb}
\makeatletter
\@setplength{toaddrvpos}{55mm}
\@setplength{toaddrheight}{25mm}
\@setplength{sigbeforevskip}{0mm}
\@setplength{firstheadwidth}{\textwidth}
\makeatother

\renewcommand*{\raggedsignature}{\raggedright}

\begin{document}

\begin{letter}{{{ recipient }}}

\setkomavar{fromname}{Referendum Stop BÜPF}
\setkomavar{fromaddress}{Röschibachstrasse 26 \\ 8037 Zürich \\ Tel. 079 7754555 \\E-Mail: ok@stopbuepf.ch}

\setkomavar{backaddress}{Referendum Stop BÜPF, 8037 Zürich}

\setkomavar{subject}{MAHNUNG - Stimmrechtsbescheinigung: Eidgenössisches Referendum gegen das Bundesgesetz betreffend die Überwachung des Post- und Fernmeldeverkehrs (BÜPF) }

\setkomavar{signature}{
\includegraphics[width=5cm]{sigJAN} \\
\vspace{3mm}
Eidg. Referendumskomitee \\
Referendum Stop BÜPF \\
Jorgo Ananiadis
}

\opening{Sehr geehrte Damen und Herren }

Am {{ sentDate }} haben wir Ihnen das beiliegende Schreiben und die entsprechende Anzahl Unterschriftenlisten mit der angegebenen Anzahl Unterschriften zur Erteilung der Stimmrechtsbescheinigung eingereicht (Kopie in der Beilage). Bei der Durchsicht unserer Liste über die ausstehenden Rücksendungen haben wir festgestellt, dass wir für die Ihnen zugestellten Unterschriftenlisten noch keine Stimmrechtsbescheinigung erhalten haben. 
 
Nach Artikel 62 Absatz 2 des Bundesgesetzes vom 17. Dezember 1976 über die politischen Rechte hat die Amtsstelle die Stimmrechtsbescheinigung auszustellen und «die Listen unverzüglich den Absendern» zurückzugeben. Bis heute haben wir von Ihnen leider keinerlei Nachricht erhalten.
 
Wir appellieren an Sie, Ihre gesetzlichen Obliegenheiten korrekt zu erfüllen, und ersuchen Sie ein letztes Mal, die Unterschriften unverzüglich bescheinigt zurückzusenden an: 

\leftskip=3mm
Referendum Stop BÜPF \\
Röschibachstrasse 26 \\
8037 Zürich

\leftskip=0mm
Wir machen Sie darauf aufmerksam, dass die Referendumsfrist am 7.7.2016 abläuft. Eine Rechtsverzögerungsbeschwerde an Ihre Aufsichtsinstanz dürfte nicht nur uns, sondern auch Ihnen erheblich mehr Aufwand verursachen als die gesetzeskonforme Erfüllung Ihrer Pflichten.

Sollte sich unsere Mahnung mit Ihrer Rücksendung gekreuzt haben, so wollen Sie dieses Schreiben bitte als gegenstandslos betrachten. 

\closing{Mit bestem Dank für Ihr Verständnis und Ihre Bemühungen und mit freundlichen Grüssen.}

\end{letter}
\end{document}

