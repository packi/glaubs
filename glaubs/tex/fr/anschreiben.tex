\documentclass[a4paper,10pt,parskip=half,french]{scrlttr2}
\usepackage[utf8]{inputenc}
\usepackage{babel}
\usepackage{fullpage}
\usepackage{graphicx}

\LoadLetterOption{SN}
\setlength{\parindent}{0pt}
\KOMAoptions{foldmarks=hb}
\makeatletter
\@setplength{toaddrvpos}{55mm}
\@setplength{toaddrhpos}{120mm}
\@setplength{toaddrheight}{25mm}
\@setplength{sigbeforevskip}{0mm}
\@setplength{firstheadwidth}{\textwidth}
\makeatother

\renewcommand*{\raggedsignature}{\raggedright}

\begin{document}

\begin{letter}{ {{ recipient }} }

\setkomavar{fromname}{Referendum Stop BÜPF}
\setkomavar{fromaddress}{Röschibachstrasse 26 \\ 8037 Zürich \\ Tel. 079 7754555 \\E-Mail: ok@stopbuepf.ch}

\setkomavar{backaddress}{Referendum Stop BÜPF, 8037 Zürich}

\setkomavar{subject}{Attestation de la qualité d’électeur: Référendum fédéral contre la loi fédérale sur la surveillance de la correspondance par poste et télécommunication (LSCPT)}

\setkomavar{signature}{
\includegraphics[width=5cm]{sigJAN} \\
\vspace{3mm}
COMITÉ RÉFÉRENDAIRE FÉDÉRAL \\
Referendum Stop BÜPF \\
Jorgo Ananiadis
}

\opening{Madame, Monsieur,}

Nous fondant sur les articles 62 et 63 de la loi fédérale du 17 décembre 1976 sur les droits politiques, nous vous remettons ci-joint {{ listCount }} listes de signatures (Nos  {{ listMin }} à {{ listMax }}) comprenant au total {{ sigCount }} signatures à l’appui de notre référendum fédéral contre la

\leftskip=3mm
Loi fédérale sur la surveillance de la correspondance par poste et télécommunication (LSCPT)

\leftskip=0mm
Nous vous prions de bien vouloir attester le droit de vote des signataires. Nous vous serions reconnaissants de renvoyer, d’ici au {{ dueDate }} au plus tard, les listes de signatures à l’adresse suivante :

\leftskip=3mm
Referendum Stop BÜPF \\
Röschibachstrasse 26 \\
8037 Zürich

\leftskip=0mm
\closing{En vous en remerciant par avance, nous vous prions d’agréer, Madame, Monsieur, l’assurance de notre considération distinguée.}

\end{letter}
\end{document}

